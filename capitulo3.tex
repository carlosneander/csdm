\chapter{Configura\c c\~oes avan\c cadas}

Este cap\'itulo tamb\'em concentra recomenda\c c\~oes de seguran\c ca que dizem respeito \`a interface do usu\'ario. Por\'em, devido \`as suas caracter\'isticas peculiares, destinam-se a dispositivos Android nos quais a seguran\c ca \'e primordial. Estas recomenda\c c\~oes podem impactar significamente a usabilidade do dispositivo Android, por isso \'e recomendado consider\'a-las como medidas de defesa em profundidade.

\section{Criar uma senha alfanum\'erica}

Esta recomenda\c c\~ao apenas sugere que o leitor crie uma senha composta n\~ao de n\'umeros, de letras, e s\'imbolos especiais para desbloquear a tela do dispositivo Android.

Esta Configurar \'e bem simples. Basta tomar este cuidado ao criar ou alterar uma senha, no passo 8 da se\c c\~ao 2.2.

A digita\c c\~ao frequente de uma senha mais complexa ser\'a, com certeza, tediosa e complicada. Por isso, ser\'a necess\'ario ponderar entre o n\'ivel de seguran\c ca desejado e a frequ\^encia de uso do dispositivo, antes da cria\c c\~ao da senha.

\section{Desabilitar a depura\c c\~ao via USB}

A depura\c c\~ao via USB \'e extremamente \'util... Para os desenvolvedores de aplicativos Android. Ela permite que os desenvolvedores alterem o comportamento padr\~ao do dispositivo, enviem comandos para o mesmo, e acessem as informa\c c\~oes armazenadas. 

\'E recomend\'avel desabilitar este recurso, pois na maioria dos dispositivos Android, a mesma entrada f\'isica usada para acessar informa\c c\~oes, \'e usada tamb\'em para recarregar a bateria. Assim, manter habilitadas as fun\c c\~oes de dados e de comandos, aumenta a probabilidade de ataque ao dispositivo.

\begin{enumerate}
\item Pressionar o bot\~ao Aplicativos
\item Pressionar Configurar
\item Deslizar at\'e a se\c c\~ao Sistema
\item Pressionar Programador
\item Desmarcar a caixa Depura\c c\~ao USB
\item Na sub-se\c c\~ao Depura\c c\~ao, desmarcar a caixa Depura\c c\~ao USB
\end{enumerate}

A op\c c\~ao Programador s\'o ser\'a vis\'ivel se o usu\'ario habilitar explicitamente os recursos de desenvolvimento. Isso \'e feito pressionando o n\'umero da vers\~ao do Android sete vezes seguidas. Caso o usu\'ario n\~ao tenha interesse no desenvolvimento de aplicativos para Android, conv\'em n\~ao habilitar os recursos de desenvolvimento.

\section{Remover as redes Wi-Fi j\'a acessadas}

Esta recomenda\c c\~ao faz com que o dispositivo Android esque\c ca redes Wi-Fi que j\'a foram acessadas pelo usu\'ario. 

Uma rede Wi-Fi confi\'avel sempre ser\'a pass\'ivel de fraude, e sempre existir\'a o risco de dispositivo se conectar a ela caso o usu\'ario a mantenha cadastrada. Al\'em disso, caso esta rede possua um nome padr\~ao (como ``default'' ou ``D-Link'' por exemplo), a probabilidade de que o dispositivo conecte-se a ela automaticamente aumenta ainda mais.

\begin{enumerate}
\item Pressionar o bot\~ao Aplicativos
\item Pressionar Wi-Fi
\item Exibir o menu de contexto, pressionando \vdots
\item Selecionar Redes salvas
\item Pressionar as redes a serem esquecidas e logo ap\'os, pressionar Esquecer
\end{enumerate}

\section{Desabilitar todo o recurso de redes Wi-Fi}

Em ambientes onde a seguran\c ca \'e prioridade, recomenda-se que o recurso de conex\~ao a redes Wi-Fi permane\c ca desabilitado no dispositivo Android. Caso ele possua acesso a servi\c cos de dados celulares (3G ou 4G por exemplo), a conex\~ao \`a internet dever\'a ocorrer atrav\'es destas redes.

\begin{enumerate}
\item Pressionar o bot\~ao Aplicativos
\item Pressionar Configurar
\item Pressionar Wi-Fi
\item Desativar o recurso Wi-Fi
\end{enumerate}

\section{Desabilitar o servi\c co de localiza\c c\~ao}

O servi\c co de localiza\c c\~ao permite que alguns aplicativos instalados no dispositivo obtenham e usem informa\c c\~oes que indiquem a localiza\c c\~ao f\'isica do usu\'ario. Esta localiza\c c\~ao \'e determinada atrav\'es do GPS do dispositivo, da rede celular 3G ou 4G, e de redes Wi-Fi. 
Se o usu\'ario desativar os servi\c cos de localiza\c c\~ao, ele receber\'a do solicita\c c\~oes para reativ\'a-la, sempre que algum aplicativo quiser fazer uso deste recurso.

Manter o servi\c co de localiza\c c\~ao ativado aumenta a capacidade de pessoas mal intencionadas, com consider\'avel conhecimento t\'ecnico, rastrearem a localiza\c c\~ao do usu\'ario atrav\'es de sites web, aplicativos, etc.

\begin{enumerate}
\item Pressionar o bot\~ao Aplicativos
\item Pressionar Configurar
\item Deslizar at\'e a se\c c\~ao Pessoais
\item Pressionar Localiza\c c\~ao
\item Desativar o servi\c co de localiza\c c\~ao
\end{enumerate}

\section{Habilitar o Modo Avi\~ao}

Quando est\'a habilitado, o Modo Avi\~ao desativa todos os transmissores e receptores de sinais de r\'adio do dispositivo. Alguns servi\c cos desativados, s\~ao:

\begin{itemize}
\item Envio e recebimento de liga\c c\~oes
\item Envio e recebimento de SMS e MMS
\item Dados m\'oveis (3G, 4G)
\item GPS
\item Wi-Fi
\item Bluetooth
\end{itemize}

Assim, quando estas funcionalidades forem desnecess\'arias, \'e recomendado manter o dispositivo no Modo Avi\~ao. Caso a transmiss\~ao e recep\c c\~ao de sinais permane\c cam habilitados mesmo sem necessidade, haver\'a um aumento da possibilidade de que estes sinais de r\'adio sejam usados como um  meio para se atacar remotamente o dispositivo.

\begin{enumerate}
\item Pressionar o bot\~ao Aplicativos
\item Pressionar Configurar
\item Na se\c c\~ao Redes, pressionar Mais
\item Pressionar Modo avi\~ao
\end{enumerate}

\section{Desabilitar a exibi\c c\~ao de notifica\c c\~oes}

Esta recomenda\c c\~ao determina que absolutamente nenhuma notifica\c c\~ao seja exibida na tela do dispositivo, quando a mesma encontrar-se bloqueada. 

Assim, caso o dispositivo seja perdido, roubado ou furtado, pessoas com interesses maliciosos n\~ao poder\~ao obter informa\c c\~oes confidenciais a partir das notifica\c c\~oes exibidas na tela bloqueada do dispositivo.

\begin{enumerate}
\item Pressionar o bot\~ao Aplicativos
\item Pressionar Configurar 
\item Deslizar at\'e a se\c c\~ao Dispositivo
\item Pressionar Som e notifica\c c\~ao
\item Na se\c c\~ao Notifica\c c\~ao, pressionar Com o dispositivo bloqueado
\item Pressionar a op\c c\~ao N\~ao mostrar notifica\c c\~oes
\end{enumerate}

\section{Limitar a quantidade de mensagens SMS e MMS}

Esta configura\c c\~ao determina a quantidade de mensagens, por conversa, que devem permanecer armazenadas no dispositivo. Quando o limite configurado \'e atingido, as mensagens mais antigas ser\~ao apagadas, caso o dispositivo esteja devidamente configurado.

Caso o dispositivo seja comprometido de alguma forma, o impacto do vazamento de informa\c c\~ao, ser\'a menor, caso a quantidade de mensagens armazenadas seja pequena.

\begin{enumerate}
\item Pressionar o \'icone Mensagens
\item Exibir o menu de contexto, pressionando \vdots 
\item Pressionar Configura\c c\~oes
\item Marcar a caixa Excluir mensagens antigas
\item Para configurar o limite de SMS, executar os seguintes passos:
  \begin{enumerate}
  \item Pressionar Limite de mensagens de texto
  \item Digitar o limite de 100 mensagens
  \item Pressionar Definir
  \end{enumerate}
\item E para configurar o limite de MMS, executar os seguintes passos:
  \begin{enumerate}
  \item Pressionar Limite de mensagens multimidia
  \item Digitar o limite de 60 mensagens
  \item Pressionar Definir
\end{enumerate}

\end{enumerate}

\section{Habilitar o Android Device Manager}

Esta recomenda\c c\~ao, se seguida, facilitar\'a a busca e a recupera\c c\~ao de dispositivos Android roubados, furtados, ou perdidos. 

Existem v\'arios aplicativos na Play Store, com funcionalidade similar. Aqui, \'e dada prefer\^encia ao Android Device Manager, que \'e uma solu\c c\~ao nativa da plataforma Android.

Cabe aqui um aviso importante. Em caso de roubo ou furto, recomenda-se que o usu\'ario n\~ao tente recuperar o dispositivo sozinho, mas sim com o aux\'ilio da for\c ca policial.

Mais um aviso. Para que o Android Device Manager seja o mais efetivo poss\'ivel, ser\'a necess\'ario que os recursos de dados m\'oveis (3G, e 4G), Wi-Fi, e os servi\c cos de localiza\c c\~ao, permane\c cam sempre habilitados, e este pr\'e-requisito contradiz algumas recomenda\c c\~oes anteriores. Por\'em, n\~ao habilitar o recurso diminuir\'a significativamente a possibilidade de se recuperar o dispositivo. Assim, cabe ao usu\'ario ponderar sobre quais recomenda\c c\~oes s\~ao mais importantes para ele.

\begin{enumerate}
\item Pressionar o bot\~ao Aplicativos
\item Pressionar Configurar
\item Deslizar at\'e a se\c c\~ao Pessoais
\item Pressionar Seguran\c ca
\item Na secao Administra\c c\~ao do dispositivo, pressionar Selecionar administradores
\item Marcar a caixa Gerenciador de dispositivos Android
\end{enumerate}

Quando o Android Device Manager est\'a configurado, pode-se gerenciar o dispositivo remotamente de duas formas, em caso de perda, roubo, ou furto: 

\begin{itemize}
\item acessando o endere\c co https://www.google.com/android/devicemanager
\item instalando o aplicativo Android Device Manager em algum outro dispositivo Android
\end{itemize}

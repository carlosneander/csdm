\chapter{Interface do Usu\'ario}

Este cap\'itulo concentra recomenda\c c\~oes de seguran\c ca que dizem respeito \`a interface do usu\'ario. Como mencionado no pref\'acio, estas recomenda\c c\~oes possuem como caracter\'isticas, serem pr\'aticas e prudentes, fornecerem um claro benef\'icio em rela\c c\~ao \`a seguran\c ca, e gerarem um impacto m\'inimo na usabilidade do dispositivo Android.

\section{Atualizar o dispositivo para a vers\~ao mais recente do Android}

Seguir esta recomenda\c c\~ao garante que a vers\~ao do sistema operacional Android instalada no dispositivo seja sempre a mais recente. As vers\~oes atualizadas geralmente trazem consigo corre\c c\~oes para falhas cr\'iticas de seguran\c ca. 

Assim, manter o sistema Android sempre atualizado reduz a probabilidade de pessoas mal intencionadas e com compet\^encia t\'ecnica, explorarem remotamente vulnerabilidades presentes no dispositivo. 

\begin{enumerate}
\item Pressionar o bot\~ao Aplicativos
\item Pressionar Configurar
\item Deslizar at\'e a se\c c\~ao Sistema
\item Pressionar Sobre o telefone
\item Pressionar Atualiza\c c\~oes do Sistema
\end{enumerate}

Vale lembrar que, dependendo do modelo e do fabricante do dispositivo Android, atualiza\c c\~oes do sistema podem n\~ao estar dispon\'iveis. Para se resguardar dessa possibilidade, recomenda-se:

\begin{itemize} 
\item verificar se o dispositivo possui uma agenda de atualiza\c c\~oes no site do fabricante e/ou em sites especializados, antes de adquiri-lo
\item o site do fabricante e outros sites especializados tamb\'em constituem uma fonte de consulta v\'alida, caso o dispositivo j\'a tenha sido adquirido
\item contactar a operadora de telefonia tamb\'em pode ser uma op\c c\~ao, para obter detalhes sobre a atualiza\c c\~ao do sistema Android
\end{itemize} 

\section{Habilitar o bloqueio do dispositivo atrav\'es de senha} \label{senha}

Esta recomenda\c c\~ao determina que uma senha seja sempre solicitada antes de se permitir o acesso ao dispositivo. \'E altamente recomendado que uma senha seja configurada. Obviamente, a falta de uma senha diminui o esfor\c co para acessar os dados armazenados no dispositivo.

\begin{enumerate}
\item Pressionar Configurar
\item Deslizar at\'e se\c c\~ao Pessoais
\item Pressionar Seguran\c ca
\item Pressionar Bloqueio de Tela
\item Pressionar Senha
\item Digitar uma senha, e pressionar Pr\'oximo, para confirm\'a-la.
\item Confirmar a senha, e pressionar Pr\'oximo
\end{enumerate}

\section{Configurar o modo de espera da tela}

Esta recomenda\c c\~ao define a quantidade de minutos em que o dispositivo pode ficar inativo antes de requerer a senha novamente. Claro que, quanto menor o tempo, menor ser\'a a probabilidade de pessoas mal intencionadas acessarem informa\c c\~oes sem a necessidade de se digitar uma senha. O tempo recomendado \'e de no m\'aximo, dois minutos.

\begin{enumerate}
\item Pressionar o bot\~ao Aplicativos
\item Pressionar Configurar
\item Deslizar at\'e a se\c c\~ao Dispositivo
\item Pressionar Tela
\item Pressionar Modo de espera
\item Selecionar a op\c c\~ao 2 minutos, ou um per\'iodo menor
\end{enumerate}

\section{Desabilitar o recurso Notifica\c c\~ao de Redes}

Este \'e um recurso do sistema Android que orienta o dispositivo a procurar por uma rede Wi-Fi, quando o usu\'ario tenta acessar a internet e ele n\~ao se encontra na faixa de uma rede previamente usada. Quando est\'a ativada e uma nova rede encontra-se dispon\'ivel, um \'icone surgir\'a na barra de status do dispositivo, que por sua vez exibir\'a uma lista de redes dispon\'iveis.

Qual o problema em manter a notifica\c c\~ao de redes ativada? Ela aumenta o risco de o dispositivo conectar-se inadvertidamente a uma rede n\~ao confi\'avel. Isso pode ocorrer caso tal rede possua o mesmo nome de uma confi\'avel previamente usada.

\begin{enumerate}
\item Pressionar o bot\~ao Aplicativos
\item Pressionar Configurar
\item Deslizar at\'e a se\c c\~ao Configura\c c\~oes de redes
\item Pressionar Wi-Fi
\item Exibir o menu de contexto, pressionando \vdots
\item Pressionar Avan\c cado
\item Deslizar controle de Notifica\c c\~ao de rede para a posi\c c\~ao Desativado
\end{enumerate}

\section{Desabilitar o Bluetooth}

A tecnologia Bluetooth permite a conex\~ao de diversos acess\'orios ao dispositivo (fones de ouvido, kits veiculares, teclados, e outros) sem a necessidade de fios. \'E recomendado que tal recurso permane\c ca desativado quando n\~ao estiver em uso, caso contrario haver\'a um aumento do risco de descoberta do dispositivo e de conex\~ao a servi\c cos desconhecidos e n\~ao confi\'aveis baseados nesta tecnologia.

\begin{enumerate}
\item Pressionar o bot\~ao Aplicativos
\item Pressionar Configurar
\item Deslizar at\'e a se\c c\~ao Configura\c c\~oes de redes
\item Pressionar Bluetooth
\item Deslizar controle de Bluetooth para a posi\c c\~ao Desativado
\end{enumerate}

\section{Apagar as informa\c c\~oes armazenadas no dispositivo antes de se desfazer dele}

Recomenda-se apagar todas as informa\c c\~oes contidas no armazenamento interno do dispositivo, restaurando-o para as configura\c c\~oes padr\~oes de f\'abrica, antes de se desfazer do dispositivo. Algumas poss\'iveis situa\c c\~oes, s\~ao:

\begin{itemize}
\item entregar o aparelho para a assist\^encia t\'ecnica, para conserto
\item vend\^e-lo para outra pessoa
\item do\'a-lo a algu\'em
\item jog\'a-lo diretamente no lixo
\end{itemize}

Manter informa\c c\~oes pessoais no dispositivo antes de repass\'a-lo, aumenta o risco de pessoas maliciosas acessarem e publicarem informa\c c\~oes confidenciais previamente armazenadas. Esta tem sido uma das principais causas de muitos vazamentos de fotos e v\'ideos \'intimos na internet. 

Por fim, recomenda-se a realiza\c c\~ao de c\'opias de seguran\c ca (backups) das informa\c c\~oes mais importantes, antes de se realizar a remo\c c\~ao das informa\c c\~oes.

\begin{enumerate}
\item Pressionar o bot\~ao Aplicativos
\item Pressionar Configurar
\item Deslizar at\'e a se\c c\~ao Pessoais
\item Pressionar Fazer backup e redefinir
\item Pressionar Restaurar dados de f\'abrica
\item Pressionar RESTAURAR TELEFONE
\item Digitar a senha, caso seja necess\'ario
\item Pressionar Pr\'oximo
\item Pressionar Restaurar
\end{enumerate}

\section{Bloquear o cart\~ao SIM}

O cart\~ao SIM, conhecido como chip da operadora de telefonia celular, permite a realiza\c c\~ao de liga\c c\~oes telef\^onicas, al\'em do armazenamento de informa\c c\~oes de contatos, e de outras informa\c c\~oes pessoais. Esta recomenda\c c\~ao faz com que o dispositivo solicite um PIN (n\'umero pessoal de identifica\c c\~ao) para que o conte\'udo armazenado no chip possa ser acessado. Do contr\'ario, outras pessoas, al\'em do propriet\'ario do chip, poder\~ao acessar seu conte\'udo, bem como utiliz\'a-lo em outros dispositivos.

\begin{enumerate}
\item Pressionar o bot\~ao Aplicativos
\item Pressionar Configurar
\item Deslizar at\'e a se\c c\~ao Pessoais
\item Pressionar Seguran\c ca
\item Pressionar Configurar bloqueio do SIM
\item Pressionar Bloquear cart\~ao SIM
\item Pressionar OK, entendi
\item Digitar o PIN antigo
\item Pressionar OK.
\item Digitar o novo PIN do cart\~ao SIM
\item Pressionar OK
\item Redigitar o novo PIN do cart\~ao SIM
\item Pressionar OK
\end{enumerate}

Duas ressalvas a respeito dessa recomenda\c c\~ao. Primeiro, o chip possui um PIN padr\~ao da operadora de telefonia. Segundo, digitar v\'arias vezes incorretamente o PIN, bloquear\'a o chip. Para habilit\'a-lo novamente, obtenha o PUK, que \'e o c\'odigo de desbloqueio do PIN. Tanto o PIN padr\~ao quanto o PUK constam do manual do chip, quando o mesmo \'e adquirido na operadora da telefonia celular.

\section{Desabilitar a visualiza\c c\~ao de senhas}

Esta configura\c c\~ao faz com que as senhas do usu\'ario n\~ao sejam exibidas, \`a medida em que elas s\~ao digitadas no dispositivo. A justificativa \' e que sempre existe a possibilidade de uma pessoa mal intencionada observar a senha que est\'a sendo digitada, ou fragmentos dela, podendo assim adivinhar o restante da senha.

\begin{enumerate}
\item Pressionar o bot\~ao Aplicativos
\item Pressionar Configurar
\item Deslizar at\'e a se\c c\~ao Pessoais
\item Pressionar Seguran\c ca
\item Deslizar controle de Tornar as senhas vis\'iveis c	
\end{enumerate}

\section{Criptografar o dispositivo}

Dispositivos m\'oveis em geral cont\^em senhas e outras credenciais que habilitam pessoas mal intencionadas a recuperarem informa\c c\~oes confidenciais de outros recursos com os quais o dispositivo interage. Criptografar todo o conte\'udo do dispositivo diminuir\'a tal amea\c ca. Do contr\'ario, informa\c c\~oes confidenciais armazenadas no dispositivo poder\~ao ser facilmente obtidas, atrav\'es de uma grande variedade de m\'etodos.

\begin{enumerate}
\item Pressionar o bot\~ao Aplicativos
\item Pressionar Configurar
\item Deslizar at\'e a se\c c\~ao Pessoais
\item Pressionar Seguran\c ca
\item Pressionar Codificar telefone, ou Codificar tablet
\item Ler atentamente as instru\c c\~oes e reassalvas exibidas pelo dispositivo
\item Caso o usu\'ario queira continuar com a codifica\c c\~ao, pressionar CODIFICAR TELEFONE, ou CODIFICAR TABLET
\item Pressionar Continuar
\item Pressionar novamente Codificar dispositivo, ou Codificar tablet
\end{enumerate}

H\'a duas ressalvas. Primeiro, o processo \'e demorado, requer que o dispositivo esteja com a bateria completamente carregada, e permane\c ca ligado na tomada. Caso o processo seja interrompido, informa\c c\~oes poder\~ao ser perdidas. E segundo, uma vez que o processo termine, o dispositivo exigir\'a um PIN ou uma senha previamente configurada, sempre que o mesmo for ligado.

\section{Desabilitar a instala\c c\~ao de aplicativos a partir de fontes desconhecidas}

Esta recomenda\c c\~ao sugere que aplicativos sejam instalados apenas a partir da loja oficial do Google, a Google Play. Instalar aplicativos a partir de diferentes sites e outras alternativas n\~ao confi\'aveis, aumenta o risco de instala\c c\~ao (inadvertida ou n\~ao) de aplicativos maliciosos.

\begin{enumerate}
\item Pressionar o bot\~ao Aplicativos
\item Pressionar Configurar
\item Deslizar at\'e a se\c c\~ao Pessoais
\item Pressionar Seguran\c ca
\item Na se\c c\~ao Administra\c c\~ao do dispositivo, deslizar o controle de Fontes desconhecidas para a posi\c c\~ao desativado
\end{enumerate}

\chapter*{Pref\'acio}

Este pequeno livro detalha um conjunto de recomenda\c c\~oes que visam tornar mais seguros os smartphones e tablets que executam o sistema operacional Android ou iOS. O objetivo dessas recomenda\c c\~oes \'e tornar estes aparelhos mais seguros contra acesso indevido, vazamento de informa\c c\~oes confidenciais, etc. A vers\~ao do sistema operacional Android que ser\'a tratada neste trabalho \'e a 6.0, tamb\'em conhecida como Marshmallow. E a vers\~ao do sistema operacional iOS ser\'a a 10.0

Este trabalho foi originalmente concebido para ser lido pelo maior n\'umero de pessoas poss\'ivel, desde como administradores de sistemas at\'e usu\'arios finais curiosos e entusiastas da \'area de seguran\c ca ou dos sistemas operacionais m\'oveis.

\section*{Como este livro \'e organizado}

O livro \'e composto por duas partes - Android e iOS - com quatro cap\'itulos cada, que abrangem desde recomenda\c c\~oes b\'asicas at\'e as mais espec\'ificas e avan\c cadas. Segue logo abaixo uma descri\c c\~ao desta categoriza\c c\~ao. 

As recomenda\c c\~oes b\'asicas s\~ao pr\'aticas e prudentes, fornecem um claro benef\'icio em rela\c c\~ao \`a seguran\c ca, e geram um impacto m\'inimo na usabilidade do dispositivo m\'ovel, seja ele Android ou iOS. 

J\'a as recomenda\c c\~oes avan\c cadas destinam-se a dispositivos nos quais a seguran\c ca \'e primordial, atuam em sua maioria como medidas de defesa em profundidade, e podem impactar significamente a usabilidade do dispositivo.

Em cada cap\'itulo, as recomenda\c c\~oes de seguran\c ca s\~ao apresentadas textualmente, e separadas em itens. Cada item \'e composto por:

\begin{itemize}
\item um t\'itulo
\item uma breve descri\c c\~ao, detalhando o item e seu prop\'osito
\item instru\c c\~oes detalhadas, explicando como realizar a configura\c c\~ao a fim de que o prop\'osito do item seja alcan\c cado
\end{itemize}

\section*{Por que \LaTeX?}

Caso o leitor tenha visualizado os arquivos utilizados para se criar este documento, ter\'a percebido que eles n\~ao foram escritos em Word, LibreOffice, ou HTML. Eles foram escritos utilizando-se o \LaTeX\
 
Para quem n\~ao conhece, o \LaTeX\ (Lamport TeX) \'e um processador de textos e uma linguagem de marca\c c\~ao de documentos criado por um cientista da computa\c c\~ao americano chamado Leslie Lamport. Ele se baseou em um outro processador de textos, chamado TeX, este criado por um cientista da computa\c c\~ao americano chamado Donald Knuth. 

Inicialmente criado para a elabora\c c\~ao de textos acad\^emicos da \'area de ci\^encias exatas, hoje em dia o \LaTeX\ \'e usado na \'area econ\^omica e at\'e pol\'itica.

Decidiu-se escrever este trabalho em \LaTeX\ devido a algumas vantagens que ele oferece, tais como:

\begin{itemize}
\item textos em \LaTeX\ s\~ao escritos em texto plano, mas a ferramenta possui compiladores de convers\~ao para diversos outros formatos: DVI, PDF, HTML, RTF, etc.
\item ele possui suporte \`a modulariza\c c\~ao do texto, o que permite escrever cap\'itulos e se\c c\~oes em arquivos fisicamente separados
\item pode-se editar textos escritos em \LaTeX\ em diversos softwares, desde o vim (Linux) at\'e o TeXnicCenter (Windows)
\item n\~ao importa qual a plataforma (Windows, Linux, Mac OS, etc.) utilizada, o texto resultante ser\'a sempre o mesmo
\end{itemize}

Existem, por\'em, algumas situa\c c\~oes em que o usar o \LaTeX\ pode se tornar desvantajoso. Para escrever documentos que necessitam de recursos visuais mais sofisticados, por exemplo, talvez o \LaTeX\ n\~ao seja a melhor op\c c\~ao.
